\documentclass[dcc,uchile]{fcfmcourse}
\usepackage{teoria}
\usepackage[utf8x]{inputenc}
\usepackage{amsmath}
\usepackage{amsfonts,setspace}
\usepackage{caption}
\usepackage{listings}
\usepackage{hyperref}
\usepackage{color}
\usepackage{soul}
\usepackage{emojis}
\usepackage[spanish]{babel}
\usepackage[lined,boxed,commentsnumbered]{algorithm2e}
\SetKwInput{KwIn}{Input}
\renewcommand{\algorithmcfname}{Algoritmo}
\renewcommand{\figurename}{Figura}
\renewcommand{\tablename}{Tabla}


\definecolor{pblue}{rgb}{0.13,0.13,1}
\definecolor{pgreen}{rgb}{0,0.5,0}
\definecolor{porange}{rgb}{0.9,0.5,0}
\definecolor{pgrey}{rgb}{0.46,0.45,0.48}

\lstset{language=Java,
  showspaces=false,
  showtabs=false,
  breaklines=true,
  showstringspaces=false,
  breakatwhitespace=true,
  commentstyle=\color{porange},
  keywordstyle=\color{pblue},
  stringstyle=\color{pgreen},
  basicstyle=\ttfamily,
  moredelim=[il][\textcolor{pgrey}]{$ $},
  moredelim=[is][\textcolor{pgrey}]{\%\%}{\%\%}
}

\newenvironment{codebox} {\small \ttfamily \obeylines \begingroup \setstretch{-2.4}} {\endgroup}

% COmpletar titulo
\title{Tarea 1 - Diccionarios en Memoria Secundaria}
\course[CC4102]{Diseño y Análisis de Algoritmos}
\professor{Pablo Barceló}
\professor{Gonzalo Navarro}
\assistant{Manuel Cáceres}
\assistantt{Fabián Mosso}
\assistantt{Jaime Salas}


\begin{document}
\captionsetup[table]{name=Tabla}
\captionsetup[table]{name=Figura}

\maketitle
\vspace{-1ex}
\section{Introducción}
El objetivo de esta tarea es implementar y evaluar en la práctica diferentes formas de implementar diccionarios en memoria secundaria vistas en clases:
\begin{itemize}
    \item B-Trees
    \item Hashing Lineal
    \item Hashing Extendible
\end{itemize}

Se espera que se \textit{implementen} los algoritmos, \textit{realicen} los experimentos correspondientes y se entregue un \textit{informe} que indique claramente los siguientes puntos:
\begin{enumerate}[1.]
    \item Las \textit{hipótesis} escogidas antes de realizar los experimentos.
    \item El \textit{diseño experimental}, incluyendo los detalles de la implementación de los algoritmos, la generación de las instancias y las medidas de rendimiento utilizadas.
    \item La \textit{presentación de los resultados} en forma de una descripción textual, tablas y/o gráficos.
    \item El \textit{análisis e interpretación} de los resultados.
\end{enumerate}
\section{Implementación}
Se pide que implemente cuatro diccionarios en memoria secundaria:
\begin{enumerate}[1.]
    \item B-Trees
    \item Hashing Extendible
    \item Dos variantes de Hashing Lineal
\end{enumerate}
Las variantes de Hashing Lineal se diferencian en las políticas utilizadas para la expansión y contracción de la estructura: estas políticas deberán ser escogidas por usted. Explique claramente en su informe las políticas que escoja. \\
Escoja el parámetro $B$ (el tamaño de la unidad mínima de I/O) para sus estructuras de acuerdo a las características de su máquina. Documente las carasterísticas de su máquina, el sistema operativo, lenguaje y compilador utilizados, RAM, y características del disco duro.

\subsection*{Funciones de Hash}
En esta tarea se trabajará con cadenas de ADN. De esta forma es necesario definir una función de hash adecuada para las estructuras de Hashing Extendible y Lineal. Utilice como hash una codificación binaria de las cadenas, usando dos bits para cada base nitrogenada ($G, C, A, T$). Así, el hash es la representación binaria de la cadena misma.

\section{Experimentos y Datos}
Trabaje con los siguientes conjuntos de datos:

\subsection*{Datos aleatorios}
Genere $2^{20}$ cadenas de bases nitrogenadas (es decir, compuestas por los caracteres $G, C, A, T$) de largo $15$ de forma aleatoria.
\begin{itemize}
    \item Inserte estas cadenas en la estructura, realizando las mediciones que se indican más adelante. Luego de insertar $2^i$ elementos en la estructura, para $i \in \{15, \ldots, 20\}$, realice lo siguiente:
    \begin{itemize}
        \item Obtenga $1000$ cadenas elegidas al azar de las ya insertadas. Utilice éstas para probar búsquedas exitosas en la estructura.
        \item Obtenga $1000$ cadenas de ADN de largo $15$, generadas al azar. Utilice éstas para probar búsquedas infructuosas en la estructura.
    \end{itemize}
    \item Elimine todos los elementos de la estructura, en un orden aleatorio.
\end{itemize}

\subsection*{Mediciones}
Obtenga las siguientes medidas de rendimiento:

\subsubsection*{Ocupación de la estructura}
Mida el espacio efectivamente ocupado por la información insertada en la estructura dividido por el espacio comprendido por los bloques de disco utilizados. Mida esta cantidad:
\begin{itemize}
    \item Luego de insertar $2^i$ elementos en la estructura, con $i\in \{15,\ldots,20\}$.
    \item Al llegar a $2^i$ elementos en la estructura en el proceso de borrado, con $i\in \{15,\ldots,19\}$.
\end{itemize}

\subsubsection*{Operaciones de lectura/escritura en disco}
Mida el número de operaciones de escritura y lectura en disco en los siguientes casos:
\begin{itemize}
    \item Luego de insertar $2^i$ elementos en la estructura, con $i\in \{15,\ldots,20\}$.
    \item Al realizar las búsquedas de los $2^i$ elementos (para $i \in \{15,\ldots,20\}$) ya insertados (búsquedas exitosas).
    \item Al realizar las búsquedas de los $2^i$ elementos (para $i\in\{15\ldots, 20\}$) generados al azar (búsquedas infructuosas).
    \item De la misma forma, mida el número de operaciones de lectura y escritura a disco necesarias al llegar a $2^i$ elementos en la estructura, durante el proceso de borrado con $i\in\{15,\ldots,19\}$.
    \item Mida además, el número necesario para vaciar completamente la estructura: así, podrá determinar el número de operaciones necesarias para vaciar una estructura con $2^i$ elementos, con $i\in\{15,\ldots, 20\}$.
\end{itemize}
Determine, además, y de acuerdo a las repeticiones que haga de sus experimentos, el error asociado a los promedios para las mediciones que realice, y utilícelo en sus gráficos.

\section{Entrega de la Tarea}
\begin{itemize}
    \item La tarea puede realizarse en grupos de a lo más 2 personas.
    \item Para la implementación solo puede utilizar C, C++ o Java. Para el informe se recomienda utilizar \LaTeX .
    \item Siga buenas prácticas (\textit{good coding practices}) en sus implementaciones.
    \item Escriba un informe claro y conciso. Las ponderaciones del informe y la implementación en su nota final son las mismas.
    \item Tenga en cuenta las sugerencias realizadas en las primeras clases sobre la forma de realizar y presentar experimentos.
    \item La entrega será a través de U-Cursos y deberá incluir el informe junto con el código fuente de la implementación (y todas las indicaciones necesarias para su ejecución).
    \item Se permiten atrasos con un descuento de 1 punto por día.
\end{itemize}
\end{document}

