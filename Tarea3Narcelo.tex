\documentclass[dcc,uchile]{fcfmcourse}
\usepackage{teoria}
\usepackage[utf8x]{inputenc}
\usepackage{amsmath}
\usepackage{amsfonts,setspace}
\usepackage{caption}
\usepackage{listings}
\usepackage{hyperref}
\usepackage{color}
\usepackage{soul}
\usepackage{emojis}
\usepackage[spanish]{babel}
\usepackage[lined,boxed,commentsnumbered]{algorithm2e}
\SetKwInput{KwIn}{Input}
\renewcommand{\algorithmcfname}{Algoritmo}
\renewcommand{\figurename}{Figura}
\renewcommand{\tablename}{Tabla}


\definecolor{pblue}{rgb}{0.13,0.13,1}
\definecolor{pgreen}{rgb}{0,0.5,0}
\definecolor{porange}{rgb}{0.9,0.5,0}
\definecolor{pgrey}{rgb}{0.46,0.45,0.48}

\lstset{language=Java,
  showspaces=false,
  showtabs=false,
  breaklines=true,
  showstringspaces=false,
  breakatwhitespace=true,
  commentstyle=\color{porange},
  keywordstyle=\color{pblue},
  stringstyle=\color{pgreen},
  basicstyle=\ttfamily,
  moredelim=[il][\textcolor{pgrey}]{$ $},
  moredelim=[is][\textcolor{pgrey}]{\%\%}{\%\%}
}

\newenvironment{codebox} {\small \ttfamily \obeylines \begingroup \setstretch{-2.4}} {\endgroup}

% COmpletar titulo
\title{Tarea 3 - Min-Cut}
\course[CC4102]{Diseño y Análisis de Algoritmos}
\professor{Pablo Barceló}
\professor{Gonzalo Navarro}
\assistant{Manuel Cáceres}
\assistantt{Fabián Mosso}
\assistantt{Jaime Salas}

\begin{document}
\captionsetup[table]{name=Tabla}
\captionsetup[table]{name=Figura}

\maketitle
\vspace{-1ex}
\section{Introducción}
Dado un grafo $G = (V,E)$ simple (no dirigido, sin \textit{self-loops} ni multiaristas) y conexo, llamaremos \textit{corte} a todo conjunto de aristas $C\subseteq E$ que al sacarlas de $G$ desconecta el grafo en dos o más componentes conexas, es decir $(V,E\setminus C)$ no es conexo.\\ 
El problema de \textit{min-cut} consiste en encontrar un \textit{corte} de tamaño mínimo (pues puede que exista más de uno de tamaño mínimo).\\

En esta tarea se trabajará con tres enfoques para resolver este problema: uno determinista, uno probabilista y una mezcla de los anteriores. Se solicita que sus hipótesis y análisis se enfoquen en los \textit{tradeoffs} de correctitud (probabilidad de error) versus tiempo.\\
Se espera que se \textit{implementen} los algoritmos, \textit{realicen} los experimentos correspondientes y se entregue un \textit{informe} que indique claramente los siguientes puntos:
\begin{enumerate}[1.]
    \item Las \textit{hipótesis} escogidas antes de realizar los experimentos.
    \item El \textit{diseño experimental}, incluyendo los detalles de la implementación de los algoritmos, la generación de las instancias y las medidas de rendimiento utilizadas.
    \item La \textit{presentación de los resultados} en forma de una descripción textual, tablas y/o gráficos.
    \item El \textit{análisis e interpretación} de los resultados.
\end{enumerate}

\section*{Teorema de Weighted-Min-Cut y Max-Flow}
\textit{Min-Cut} se puede resolver determinísticamente usando una generalización del problema denominada \textit{Weighted-Min-Cut} y la relación de este último con el problema \textit{Max-Flow} de flujos en grafos.

\subsection*{Weighted-Min-Cut}
Es una generalización de \textit{Min-Cut} que considera pesos en las aristas del grafo, la idea ahora es encontrar el \textit{corte} de menor peso. Más formalmente, sea $w:E\to \mathbb{R}^+$, una función de pesos en las aristas del grafo y $C$ un \textit{corte} del grafo, definimos el peso de $C$ como la suma de los pesos de sus arcos, $w(C) = \sum_{e\in C} w(e)$. El problema de \textit{Weighted-Min-Cut} consiste en encontrar el \textit{corte} de peso mínimo.

\subsubsection*{s-t, Weighted-Min-Cut}
Es una variante del problema  \textit{Weighted-Min-Cut} que fija dos vértices $s\not = t \in V$ y solo considera aquellos \textit{cortes} que desconectan a $s$ de $t$ en el grafo.

\subsection*{Max-Flow}
El problema de \textit{Max-Flow} consiste en encontrar la mayor cantidad de \textit{flujo} que se puede enviar entre dos nodos del grafo, es por eso que se define para grafos dirigidos y se tiene como restricciones al problema cotas superiores en la cantidad de flujo que se puede enviar por una arista: la capacidad.\\ Más formalmente, dado un grafo dirigido $\vec{G} = (V, \vec{E})$, una función de capacidad sobre las aristas $w:\vec{E}\to \mathbb{R}^+$ (con su extensión respectiva para conjuntos de aristas) y dos vértices $s,t\in V, s\not = t$, se define un \textit{s-t flujo}, como una asignación de las aristas $f:\vec{E}\to \mathbb{R}^+$, tal que:
\begin{itemize}
    \item $f(u,v)\le w(u,v), \forall (u,v) \in \vec{E}$, es decir, el flujo enviado por una arista es menor o igual a su capacidad. A esta propiedad se le denomina \textit{restricción de la capacidad}.
    \item $\sum_{(u,v) \in \vec{E}} f(u,v) = \sum_{(v,u) \in \vec{E}} f(v,u), \forall v \in V\setminus\{s,t\}$, es decir, el flujo de entrada es igual al flujo de salida. A esta propiedad se le denomina \textit{conservación del flujo}.
\end{itemize}
  El problema de \textit{Max-Flow} consiste en encontrar la asignación $f$, que maximice el flujo enviado desde $s$, $\sum_{(s,v)\in\vec{E}}f(s,v)$.
\subsection*{Teorema de Weighted-Min-Cut, Max-Flow}
Este teorema expone que dados dos vértices $s,t$ en un grafo dirigido $(V,\vec{E})$, en el cual la asignación de capacidad y la de pesos es idéntica, el valor del máximo flujo enviado desde $s$ a $t$ es igual al peso del \textit{s-t, Weighted-Min-Cut}.
\section{Algoritmo Determinista}
Para determinar la solución de \textit{Min-Cut} usaremos un algoritmo $MF$ que resuelva \textit{Max-Flow} de manera eficiente \footnote{Esta implementación puede ser obtenida de una fuente externa, al final de este archivo se encuentran algunas fuentes \ref{Links}.}. La reducción es como sigue: 
\begin{itemize}
    \item El resultado de \textit{Min-Cut} lo obtendremos de \textit{Weighted-Min-Cut}, asignando a todas las aristas un peso de $1$.
    \item Para resolver \textit{Weighted-Min-Cut}, lo que haremos será utilizar el teorema \textit{Weighted-Min-Cut, Max-Flow}. Más específicamente, fijamos un $s\in V$ y luego por cada $t\not = s\in V$, usamos $MF$ para encontrar el \textit{s-t , Max-Flow} (que es igual al \textit{s-t, Weighted-Min-Cut})\footnote{El grafo de input utilizado por $MF$ es mismo del input poniendo aristas dirigidas en ambos sentidos con capacidades igual al peso de la arista del grafo original.}; finalmente la respuesta será el mínimo entre todas estas ejecuciones.
\end{itemize}

\section{Algoritmo de Karger}
El \textit{algoritmo de Karger} es un algoritmo probabilista que resuelve el problema de \textit{Min-Cut} con una probabilidad de éxito mayor a $\binom{n}{2}^{-1}$, para lograrlo transforma el grafo de entrada en un grafo con \textit{multiaristas} (pero sin \textit{self-loops}) y contrae aristas al azar del grafo, hasta que el grafo lo conforman solo $2$ vértices.\

Entenderemos \textit{contraer} una arista $e = \{u,v\} \in E$, como el proceso de fusionar los extremos de la arista en un nuevo nodo cuyos vecinos corresponden a los vecinos de los nodos fusionados, excepto ellos mismos (en caso que $u$ y $v$ sean vecinos). Más formalmente, llamaremos $G/e$ al grafo $G$ al que se le contrajo la arista $e$ dado por:
\begin{align*}
G/e =& (V\setminus \{u,v\} \cup \{v_{u,v}\} ,\\
&(e = \{x,y\} \in E : x \not = u \land y \not = u \land x \not = v \land y \not = v ) \\ & +(\{x,v_{u,v}\} : x \in \{\{y,v\} \in E\} ) \\ & +(\{x,v_{u,v}\} : x \in \{\{y,u\} \in E\} ))   
\end{align*}
Con esto el \textit{algoritmo de Karger} es como sigue :\\
\begin{algorithm}[H]
        \SetAlgoLined
        $(V',E') = Multigraph(V,E)$\\
        \For{$i\gets 1\ldots n-2$}{
            Elegimos una arista $e\in E'$ uniformemente al azar y la contraemos del grafo $(V',E')$ actualizando $V'$ y $E'$.
        }
        \Return $E'$
\end{algorithm}

Finalmente, para disminuir la probabilidad de error del algoritmo, lo que haremos será ejecutarlo $k$ veces y responder el menor conjunto de aristas encontrado como el \textit{Min-Cut}.
\section{Una mezcla}
El tercer approach que tomaremos será utilizar ambas ideas expuestas anteriormente. Esto nos permitirá ajustar el \textit{tradeoff} de tiempo versus probabilidad de error. Para esto ejecutaremos el \textit{algoritmo de Karger} pero ahora hasta que nos queden $t \in \{2, 3, \ldots, n\}$ vértices en el multigrafo y ocuparemos el algoritmo $MF$ para computar el \textit{Min-Cut} del grafo resultante, dándolo como respuesta.\textit{Note que el parámetro $t$ es un calibrador del \textit{tradeoff} de tiempo versus probabilidad de error, a mayor $t$ menor es la probabilidad de error pero el tiempo aumenta}.\\
Finalmente, consideraremos repetir este algoritmo probabilista $k$ veces para disminuir la probabilidad de error.\newpage

Con esto, nuestro algoritmo probabilista queda como sigue :\\
\begin{algorithm}[H]
        \SetAlgoLined
        $(V',E') = Multigraph(V,E)$\\
        \For{$i\gets 1\ldots n-t$}{
            Elegimos una arista $e\in E'$ uniformemente al azar y la contraemos del grafo $(V',E')$ actualizando $V'$ y $E'$.
        }
        \Return $Min-Cut-Determinista((V',E'))$
\end{algorithm}

\section{Experimentos}
En esta tarea se pide comparar la correctitud  (probabilidad de error) y el tiempo de los resultados dados por los tres algoritmos antes explicados. Para esto tendrá que correr todas las implementaciones en grafos de distintos tamaños explicados más adelante.\\

Recuerde repetir los experimentos para los distintos tamaños de manera de obtener promedios confiables, tampoco olvide documentar los intervalos de confianza de sus resultados.\\

Para \textit{Max-Flow} reporte tanto los algoritmos (nombre y complejidad teórica) utilizados por la implementación, como los tiempos correspondientes.\\

Para las tres implementaciones reporte los tiempos en los distintos inputs, en el caso de la implementación del \textit{algoritmo de Karger} reporte como avanza la probabilidad de error según el número $k$ de repeticiones realizadas. Para el algoritmo \textit{una mezcla}, realice el mismo reporte anterior considerando $5$ valores de $t$ distribuidos uniformemente en el rango $[3,n-1]$.


\subsection*{Datos}

\subsubsection*{Grafos Aleatorios}
Construya grafos de $n = 2^i$ con $i \in \{10, 11, \ldots, 15\}$ vértices y genere las aristas según el siguiente proceso probabilístico:\\

\begin{algorithm}[H]

\SetAlgoLined
\KwIn{$(V,p)$}
 $E\gets \emptyset$\;
 \For{$(u, v) \in \binom{V}{2}$}{
  Con probabilidad $p$ hacer:
    $E\gets E \cup \{(u, v)\}$\;
 }
 \KwRet{$G = (V, E)$}

 
\end{algorithm}
Escoja al menos 5 valores de $p$ distribuidos en $[1/n,  1]$, escoja los valores de $p$ de modo de poner a prueba sus hipótesis.
\section{Entrega de la Tarea}
\begin{itemize}
    \item La tarea puede realizarse en grupos de a lo más 2 personas.
    \item Para la implementación solo puede utilizar C, C++ o Java. Para el informe se recomienda utilizar \LaTeX .
    \item Siga buenas prácticas (\textit{good coding practices}) en sus implementaciones.
    \item Escriba un informe claro y conciso. Las ponderaciones del informe y la implementación en su nota final son las mismas.
    \item Tenga en cuenta las sugerencias realizadas en las primeras clases sobre la forma de realizar y presentar experimentos.
    \item La entrega será a través de U-Cursos y deberá incluir el informe junto con el código fuente de la implementación (y todas las indicaciones necesarias para su ejecución).
    \item NO se permitirán atrasos para esta tarea.
\end{itemize}
\section{Links}\label{Links}Los siguientes links contienen implementaciones del algoritmo de \textit{Max-Flow} :
\begin{itemize}
    \item La librería Boost C++ incluye una implementación de \textit{Max-Flow}, \url{https://www.boost.org/}.
    \item La libreria algs4 Java incluye una implementación de \textit{Max-Flow}, \url{https://algs4.cs.princeton.edu/code/}
\end{itemize}
\end{document}

