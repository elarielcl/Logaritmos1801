\documentclass[dcc,uchile]{fcfmcourse}
\usepackage{teoria}
\usepackage[utf8x]{inputenc}
\usepackage{amsmath, amssymb, amsthm}
\usepackage{amsfonts,setspace}
\usepackage{listings}
\usepackage{hyperref}
\usepackage{color}
\usepackage{csquotes}
\usepackage{soul}
\usepackage{emojis}
\usepackage[linesnumbered,lined,boxed,commentsnumbered]{algorithm2e}
\renewcommand{\algorithmcfname}{Algoritmo}
\renewcommand{\figurename}{Figura}
%Teoremas, Lemas, etc.
\theoremstyle{plain}
\newtheorem{teo}{Teorema}
\newtheorem{lem}{Lema}
\newtheorem{prop}{Proposici\'on}
\newtheorem{cor}{Corolario}
\theoremstyle{definition}
\newtheorem{defi}{Definici\'on}
\newtheorem{obs}{Observaci\'on}
\newtheorem{ej}{Ejemplo}
\newtheorem{ejer}{Ejercicio}

\definecolor{pblue}{rgb}{0.13,0.13,1}
\definecolor{pgreen}{rgb}{0,0.5,0}
\definecolor{porange}{rgb}{0.9,0.5,0}
\definecolor{pgrey}{rgb}{0.46,0.45,0.48}

\lstset{language=Java,
  showspaces=false,
  showtabs=false,
  breaklines=true,
  showstringspaces=false,
  breakatwhitespace=true,
  commentstyle=\color{porange},
  keywordstyle=\color{pblue},
  stringstyle=\color{pgreen},
  basicstyle=\ttfamily,
  moredelim=[il][\textcolor{pgrey}]{$ $},
  moredelim=[is][\textcolor{pgrey}]{\%\%}{\%\%}
}

\newenvironment{codebox} {\small \ttfamily \obeylines \begingroup \setstretch{-2.4}} {\endgroup}

% COmpletar titulo
\title{Auxiliar 7 - ``Búsqueda en Texto, Algoritmo basado en Autómata"}
\course[CC4102]{Diseño y Análisis de Algoritmos}
\professor{Pablo Barceló}
\professor{Gonzalo Navarro}
\assistant{\textst{Manuel} Ariel Cáceres Reyes}


\begin{document}
\maketitle
\begin{center}
7 de Mayo del 2018
\end{center}


\vspace{-1ex}

\cfoot{``I visualize a time when we will be to robots what dogs are to humans, and I’m rooting for the machines.''\\ Claude Shannon}

\begin{problems}
\problem {\large\underline{\textbf{Búsqueda en Texto}}}\\
El problema de \textit{Búsqueda en Texto} consiste en, dadas dos cadenas $T$ (texto) y $P$ (patrón) con caracteres en un alfabeto $\Sigma$, encontrar todas las ocurrencias de $P$ en $T$ (esto es, las subcadenas de $T$ iguales a $P$). \\ Parametrizaremos los costos de las soluciones por los tamaños de $T$, $P$ y $\Sigma$ que serán $n$, $m$ y $\sigma$ respectivamente.
\begin{enumerate}[a)]
    \item Muestre una solución que funcione en tiempo $\mathcal{O}(nm)$
    \item Muestre que existe un autómata finito determinista que sea capaz de reconocer $(\Sigma)*P$
    \item ¿Cómo se podría usar este autómata para resolver el problema de \textit{Búsqueda en Texto}?
    \item Muestre como construir este autómata de forma eficiente.
    \item Muestre un algoritmo de tiempo $\mathcal{O}(n+m)$ (puede considerar que $\sigma \in \mathcal{O}(1)$).
    \item Muestre un algoritmos de tiempo $\mathcal{O}(n+m)$ incluso cuando $\sigma \in \omega(1)$.

\end{enumerate}
\end{problems}
\newpage
\begin{center}
{\huge \underline{Soluciones}}
\end{center}
\begin{problems}
\problem{\large\underline{\textbf{Búsqueda en Texto}}}\\
\begin{enumerate}[a)]
    \item El algoritmo por fuerza bruta\\
    \begin{algorithm}[H]
        \SetAlgoLined
        \For{$i=0\ldots n-m$} {
            $j\gets 1$
            \For{$j=1\ldots m$} {
                \If{$P[j] != T[i+j]$}{
                    $break$
                }
            }
            \If{$j==m$} {
                $report(i+1)$
            }
        }
    \end{algorithm}
    multiplicando el costo de los ifs obtenemos un costo $\mathcal{O}(nm)$.
    \item El autómata es el siguiente:
    \begin{itemize}
    \item Conjunto de estados $Q=\{0, 1, \ldots, m\}$ siendo $0$ el estado inicial y $m$ el final.
    \item Función de transición 
    \begin{eqnarray*}
    \delta: \Sigma \times Q \to & Q & \\
    (q, a) \mapsto &\delta(q, a) &= \sigma(P[1:q]a) \\
    &  &= \text{largo del sufijo más largo de $P[1:q]a$ que es prefijo de $P$}
    \end{eqnarray*}
    \end{itemize}
\begin{figure}[h]\footnotesize
  \begin{center}
      \includegraphics[scale=0.28]{imagenes/automata.png}
  \caption{Autómata correspondiente al patrón $P=ababaca$.}
  \end{center}
\end{figure}
Veamos ahora lo siguiente:
\begin{quote}
    Si el sufijo más largo de lo que se ha
    leído que es prefijo del patrón es $P[1:i]$, entonces el autómata se encuentra en el estado $i$
\end{quote}
(esto pues justo antes de leer el último carácter $P[i]$ se estaba en el estado $j\ge i-1$, por lo que se aplicó $\sigma(P[1:j]P[i]) = \sigma(P[1:i])$ (esto último por ser $P[1:i]$ maximal) $= i$).\\
Con esto es simple ver que cada vez que nuestro prefijo maximal leído sea $P$ el autómata se encontrará en el estado $m$, reconociendo así $(\Sigma)*P$
    \item Podemos correr el autómata anterior sobre $T$ y reportar una ocurrencia cada vez que pasamos por el estado final.
    \item
    Construir el autómata se reduce a construir su función de transición. El algoritmo más simple consiste en hacerlo para cada par $(q, a)\in Q \times \Sigma$ y para cada uno de ellos revisar si $P[1:k]$ es sufijo de $P[1:q]a$ , con $k$ desde $\min(q+1, m)$ hasta $0$ incurriendo en un costo total de $\mathcal{O}(|\Sigma|m^3) = \mathcal{O}(m^3)$.\\
    
    Consideremos ahora la siguiente observación:\\
    
    Al calcular $\delta(q,a) = \sigma(P[1:q]a)$
    \begin{itemize}
        \item Si $P[q+1] = a$, entonces $\delta(q,a) = q+1$.
        \item Si no ($P[q+1] \not= a$), entonces $\delta(q,a) = \sigma(P[1:q]a) = \sigma(P[2:q]a) \le q$, y por lo tanto, puede ser calculado corriendo el autómata parcialmente construído sobre $P[2:q]a$.
    \end{itemize}
    Considerando esto podemos construir el autómata en $\mathcal{O}(|\Sigma|m^2) = \mathcal{O}(m^2)$.\\
    
    
    Finalmente, en el caso $P[q+1] \not= a$ no es necesario correr el autómata sobre $P[2:q]a$. Esto se puede lograr manteniendo una variable $X$ que sea el resultado de correr el autómata sobre $P[2:q]$, es decir, $X = \delta(\ldots\delta(\delta(0, P[2]), P[3])\ldots , P[q])$. Con esto, $\delta(q, a) = \delta(X, a)$, cuando $P[q+1] \not= a$. Obteniendo un tiempo de $\mathcal{O}(|\Sigma|m) = \mathcal{O}(m)$.
    \item Construimos el autómata de $P$ en $\mathcal{O}(m)$ y luego lo corremos en tiempo $\mathcal{O}(n)$.
    \item La solución anterior es en realidad $\mathcal{O}(n + \sigma m)$, y en caso que $\sigma \in \omega(1)$, no es $\mathcal{O}(n+m)$ . Esta complejidad se logra con el algoritmo de Morris-Pratt. Este algoritmo define la función $f(i)$ como el largo del prefijo propio más largo de $P[1:i]$ que también es sufijo propio.\\
    
    Con esto el algoritmo mantiene un contador $j$ que avanza con el texto y un contador $i$ que avanza con el patrón mientras los caracteres calcen. En caso de no haber calce, $i$ se actualiza a $f(i)$ repitiendo esto hasta que se encuentre un calce o $i=0$.\\
    
    Se puede mostrar que el algoritmo anteriormente descrito es correcto y que $f(i)$ se puede construir en $\mathcal{O}(m)$ (de un modo similar a la construcción del autómata y con un análisis amortizado parecido al que haremos a continuación).\\
    
    Veamos ahora que el algoritmo (teniendo $f$ precomputada) es $\mathcal{O}(n)$. Un análisis de peor caso nos lleva a pensar que por cada $j$, podríamos aplicar la función $f$ $\mathcal{O}(m)$ veces hasta encontrar un calce, por lo que la complejidad es $\mathcal{O}(nm)$, sin embargo, para poder aplicar la función $f$ un número $k$ de veces, antes debimos haber aumentado $i$ $k$ veces, finalmente, como $i$ no puede haber aumentado mas de $n$ veces, el número de veces que se aplica $f$ en total es $n$.
\end{enumerate}
\end{problems}
\end{document}
