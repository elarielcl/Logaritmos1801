\documentclass[dcc,uchile]{fcfmcourse}
\usepackage{teoria}
\usepackage[utf8x]{inputenc}
\usepackage{amsmath, amssymb, amsthm}
\usepackage{amsfonts,setspace}
\usepackage{listings}
\usepackage{hyperref}
\usepackage{color}
\usepackage{csquotes}
\usepackage{soul}
\usepackage{emojis}
\usepackage[linesnumbered,lined,boxed,commentsnumbered]{algorithm2e}
\renewcommand{\algorithmcfname}{Algoritmo}

%Teoremas, Lemas, etc.
\theoremstyle{plain}
\newtheorem{teo}{Teorema}
\newtheorem{lem}{Lema}
\newtheorem{prop}{Proposici\'on}
\newtheorem{cor}{Corolario}
\theoremstyle{definition}
\newtheorem{defi}{Definici\'on}
\newtheorem{obs}{Observaci\'on}
\newtheorem{ej}{Ejemplo}
\newtheorem{ejer}{Ejercicio}

\definecolor{pblue}{rgb}{0.13,0.13,1}
\definecolor{pgreen}{rgb}{0,0.5,0}
\definecolor{porange}{rgb}{0.9,0.5,0}
\definecolor{pgrey}{rgb}{0.46,0.45,0.48}

\lstset{language=Java,
  showspaces=false,
  showtabs=false,
  breaklines=true,
  showstringspaces=false,
  breakatwhitespace=true,
  commentstyle=\color{porange},
  keywordstyle=\color{pblue},
  stringstyle=\color{pgreen},
  basicstyle=\ttfamily,
  moredelim=[il][\textcolor{pgrey}]{$ $},
  moredelim=[is][\textcolor{pgrey}]{\%\%}{\%\%}
}

\newenvironment{codebox} {\small \ttfamily \obeylines \begingroup \setstretch{-2.4}} {\endgroup}

% COmpletar titulo
\title{Auxiliar 5 - ``R-Trees y Análisis Amortizado"}
\course[CC4102]{Diseño y Análisis de Algoritmos}
\professor{Pablo Barceló}
\professor{Gonzalo Navarro}
\assistant{\textst{Manuel} Ariel Cáceres Reyes}


\begin{document}
\maketitle
\begin{center}
16 de abril del 2018
\end{center}


\vspace{-1ex}

\cfoot{``I save my money, brother. I don't spend what I don't make.''\\ Joe Penny}

\begin{problems}
\problem \underline{\textbf{Quack}}\\
 Un \textit{quack} es una mezcla entre queues y stacks. Puede ser visto como una lista de elementos,
escrita de izquierda a derecha, que soporta las siguientes operaciones:
\begin{itemize}
    \item \texttt{QPush(x)} agrega x en el extremo izquierdo.
    \item \texttt{QPop(x)} remueve y retorna el elemento más a la izquierda.
    \item \texttt{QPull(x)} remueve y retorna el elemento m´as a la derecha.
\end{itemize}

Implemente un \textit{quack} usando tres stacks de modo que cada operación tenga un costo amortizado
constante, considerando que estos stacks son capaces de ejecutar las operaciones Push y Pop en tiempo constante. Almacene los elementos sólo una vez en cualquiera de los 3 stacks, y utilice a lo más $\mathcal{O}(1)$ de memoria adicional.

\problem \underline{\textbf{Trits}}\\
 Considere el siguiente sistema redundante de dígitos ternarios.
Un número se representa como una secuencia de \textit{trits}, que, como su nombre lo indica, pueden tomar tres posibles valores: $0, +1\ o\ −1$. El valor de un número $t_{k−1}, \ldots, t_0$ (donde cada $t_i$ es un \textit{trit}) se define como:
\begin{align*}
    \sum_{i=0}^{k-1} t_{i}2^i
\end{align*}

El proceso de incrementar un número de este tipo es análogo al de la operación en números binarios. Se añade $1$ al \textit{trit} menos significativo; si el resultado es $2$, se cambia
el \textit{trit} a $0$ y se propaga una reserva hacia el siguiente \textit{trit}. Se repite el proceso hasta que no exista carry.
El proceso de decrementar es análogo. Comenzando de $0$, se realiza una secuencia de $n$ incrementos y decrementos (sin un orden particular).\ 

Demuestre que, utilizando esta representación, el costo amortizado por operación es $\mathcal{O}(1)$.
\problem \underline{\textbf{R-Trees}}

\end{problems}

\newpage
\begin{center}
{\huge \underline{Soluciones}}
\end{center}
\begin{problems}
\problem \underline{\textbf{Quack}}\\
Tendremos una pila de $entrada$ en donde \texttt{qpusheamos} y \texttt{qpopeamos} los elementos, y además una pila de $salida$ de donde \texttt{qpulleamos}. En caso que la pila de $entrada$ no tenga elementos para \texttt{qpopear}, vaciamos la primera mitad de la pila de salida en una pila $aux$, vaciamos la otra mitad en la pila de $entrada$ y vaciamos la pila $aux$ en la de salida; llamaremos a esta operación \textit{refilling de entrada} y tiene un costo de $\mathcal{O}(n)$. Por otro lado, en caso que la pila de $salida$ no tenga elementos para \texttt{qpullear}, ponemos la mitad de los elementos de la pila de $entrada$ en una pila $aux$, vaciamos el resto en la pila de $salida$ y finalmente vaciamos la pila $aux$ en la $entrada$; llamaremos a esta operación \textit{refilling de salida}.
\begin{figure}[h]
    \centering
    \includegraphics[scale=0.5]{imagenes/dos_pilas_quack}
\end{figure}

Analizaremos las operaciones por el método de ``Contabilidad de Costos''. Primero veamos que las tres operaciones tienen costo $\mathcal{O}(1)$ siempre y cuando no se necesite un \textit{refilling}. Por lo tanto, las operaciones pagarán por adelantado estos \textit{refilling}; los \texttt{QPUSH} pagarán por adelantado el \textit{refilling de salida} en que pueda participar el elemento pusheado, los \texttt{QPULL} pagarán el \textit{refilling de salida} de la mitad correspondiente de elementos que quedaron en la pila de $entrada$ en el anterior \textit{refilling de entrada}, finalmente, los \texttt{QPOP} pagarán por adelantado el \textit{refilling de entrada} de los elementos correspondientes de la pila de $salida$ del anterior \textit{refilling de salida}.
\problem \underline{\textbf{Trits}}\\
Consideremos el potencial $\phi = $ cantidad de $1$s y $-1$s. Veamos que $\phi_0 = 0$ y $\phi_i \ge 0$ por ser una cantidad.\\
Analicemos ahora el costo amortizado por operación.\\

En el caso del incremento, supongamos que hay $l$ $1$s consecutivos en las posiciones menos significativas del número y que luego de ellos hay un $0$ o un $-1$. En este caso tenemos que:
\begin{align*}
    \hat{c} &= c + \Delta \phi \\
    &\le (l+1) + (-l + 1)\\
    &= 2
\end{align*}

En el caso del decremento, supongamos que hay  $l$ $-1$s consecutivos en las posiciones menos significativas del número y que luego de ellos hay un $0$ o un $1$. En este caso también tenemos que:
\begin{align*}
    \hat{c} &= c + \Delta \phi \\
    &\le (l+1) + (-l + 1)\\
    &= 2
\end{align*}
\problem \underline{\textbf{R-Trees}}\\
Para más información consultar el apunte del curso.
\end{problems}
\end{document}
